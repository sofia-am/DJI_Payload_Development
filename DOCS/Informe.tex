\documentclass[12pt]{article}
\usepackage[spanish]{babel}
\usepackage{ragged2e}
\usepackage{amsmath}
\usepackage{graphicx}
\usepackage{hyperref}
\usepackage[utf8]{inputenc}

\title{\huge Práctica Supervisada\vspace*{5cm}}

\author{Amallo, Sofía - Gil, Juan Manuel}

\date{\parbox{\linewidth}{\centering%
  Noviembre 18, 2022 \endgraf\bigskip
  \vspace*{4cm}
  Dr. Ing. Horacio A. Mendoza \hspace*{1cm} Dr. Ing. Jorge Finochietto\endgraf\medskip
  \vspace*{0.5cm}
  Laboratorio de\ Comunicaciones Digitales \endgraf
  Universidad Nacional de Córdoba}}
  
\begin{document}
\maketitle

\newpage

\section{Ficha de Práctica Supervisada}
\subsection{Datos de los Alumnos}
\raggedright
\textsl{Nombre y Apellido:}  Sofía Amallo

\textsl{N° de Matrícula:} 41279731

\textsl{Teléfono:} 3512355718

\textsl{Correo electrónico:} sofia.amallo@mi.unc.edu.ar
\vspace*{0.5cm}

\textsl{Nombre y Apellido:} Juan Manuel Gil

\textsl{N° de Matrícula:} 41592940

\textsl{Teléfono:} 3571604632

\textsl{Correo electrónico:} juan.manuel.gil@mi.unc.edu.ar

\subsection{Datos de la Institución Receptora}
\textsl{Nombre:} Laboratorio de Comunicaciones Digitales

\textsl{Dirección del Laboratorio:} Av. Vélez Sársfield 1600 CU, Cba - Argentina

\textsl{Nombre y Apellido del Supervisor Docente:} Dr. Ing. Jorge Manuel Finochietto

\textsl{Cargo que ocupa el Supervisor Docente en la Institución Receptora:} Secretario de Tecnología  Educación Virtual, Jefe de cátedra de Informática e Investigador Independiente del CONICET

\textsl{N° de Teléfono:} (+54) 351 5353800 int 29085

\textsl{Correo Electrónico:} lcd@fcefyn.unc.edu.ar

\subsection{Datos del Tutor}

\textsl{Nombre y Apellido:} Dr. Ing. Horacio A. Mendoza

\textsl{Cargo y Cátedra:} Docente e Investigador

\textsl{Teléfono:} (+54) 351 5353800 int 29085

\textsl{Correo Electrónico:} lcd@fcefyn.unc.edu.ar

\subsection{Datos de la Práctica Profesional Supervisada}

\textsl{Fecha de Inicio:} 26/09/2022

\textsl{Fecha de Finalización:} 18/11/2022

\textsl{Total de horas:} 222

\tableofcontents
\newpage

\justifying
\section{Resumen}
\setlength\parindent{24pt} La Práctica Supervisada (PS) se llevó a cabo en el Laboratorio de Comunicaciones Digitales (LCD) de la Universidad Nacional de Córdoba, Córdoba, Argentina, donde se abordó el desafío de desarrollar \textit{payloads}, también conocidas como cargas útiles, con el fin de expandir las funcionalidades del drone Matrice 300 RTK de DJI.
En ese contexto se investigaron distintas maneras de interactuar con dicho drone, siendo algunas el SDK (Software Development Kit) que brinda el fabricante que cumple el rol de una API (Application Programming Interface), también se investió acerca del
protocolo UART como solución al problema de la comunicación entre el drone y las distintas payloads.

Una de las partes mas enriquecedoras de nuestra experiencia en el laboratorio fue la dinámica de trabajo, ya que 
abordamos diferentes soluciones a medida que se presentaron nuevas problemáticas, algo propio del desarrollo con
nuevas tecnologías como es nuestro caso. 

La curva de aprendizaje en dichas tecnologías fue documentada en Notion, una plataforma de sofware orientada a 
la toma de notas, lo cual nos permitirá confeccionar de una manera concisa una llamada ``hoja de ruta"  que quede
a disposición de otros estudiantes que transiten por el LCD, con el objetivo de brindarle alguna retribución a la institución
que nos brindó herramientas de trabajo, acceso a equipos, placas e insumos para el desarrollo de nuestro trabajo
y sobre todo un espacio de aprendizaje junto a docentes y compañeros. Sin dudas nuestro paso por el Laboratorio de
Comunicaciones Digitales se constituyó en una de las experiencias fundamentales de nuestra formación en el campo profesional.

\newpage
\section{Introducción}


\end{document}