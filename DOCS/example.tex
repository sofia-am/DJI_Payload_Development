\documentclass[12pt]{article}
\usepackage[spanish]{babel}
\usepackage{amsmath}
\usepackage{graphicx}
\usepackage{hyperref}
\usepackage[utf8]{inputenc}

\title{\huge Práctica Supervisada\vspace*{5cm}}

\author{Amallo, Sofía - Gil, Juan Manuel}

\date{\parbox{\linewidth}{\centering%
  Noviembre 18, 2022 \endgraf\bigskip
  \vspace*{4cm}
  Dr. Ing. Horacio A. Mendoza \hspace*{1cm} Dr. Ing. Jorge Finochietto\endgraf\medskip
  \vspace*{0.5cm}
  Laboratorio de\ Comunicaciones Digitales \endgraf
  Universidad Nacional de Córdoba}}
  
\begin{document}
\maketitle

\newpage

\section{Ficha de Práctica Supervisada}
\subsection{Datos de los Alumnos}
\raggedright
\textsl{Nombre y Apellido:}  Sofía Amallo

\textsl{N° de Matrícula:} 41279731

\textsl{Teléfono:} 3512355718

\textsl{Correo electrónico:} sofia.amallo@mi.unc.edu.ar
\vspace*{0.5cm}

\textsl{Nombre y Apellido:} Juan Manuel Gil

\textsl{N° de Matrícula:} 41592940

\textsl{Teléfono:} 3571604632

\textsl{Correo electrónico:} juan.manuel.gil@mi.unc.edu.ar

\subsection{Datos de la Institución Receptora}
\textsl{Nombre:} Laboratorio de Comunicaciones Digitales

\textsl{Dirección del Laboratorio:} Av. Vélez Sársfield 1600 CU, Cba - Argentina

\textsl{Nombre y Apellido del Supervisor Docente:} Dr. Ing. Jorge Manuel Finochietto

\textsl{Cargo que ocupa el Supervisor Docente en la Institución Receptora:} Secretario de Tecnología  Educación Virtual, Jefe de cátedra de Informática e Investigador Independiente del CONICET

\textsl{N° de Teléfono:} (+54) 351 5353800 int 29085

\textsl{Correo Electrónico:} lcd@fcefyn.unc.edu.ar

\subsection{Datos del Tutor}

\textsl{Nombre y Apellido:} Dr. Ing. Horacio A. Mendoza

\textsl{Cargo y Cátedra:} Docente e Investigador

\textsl{Teléfono:} (+54) 351 5353800 int 29085

\textsl{Correo Electrónico:} lcd@fcefyn.unc.edu.ar

\subsection{Datos de la Práctica Profesional Supervisada}

\textsl{Fecha de Inicio:} 26/09/2022

\textsl{Fecha de Finalización:} 18/11/2022

\textsl{Total de horas:} 222

\tableofcontents
\newpage
\section{Resumen}
\setlength\parindent{24pt} La Práctica Supervisada (PS) se llevó a cabo en el Laboratorio de Comunicaciones Digitales (LCD) de la Universidad Nacional de Córdoba, Córdoba, Argentina, donde se abordó el desafío de desarrollar \textit{payloads}, también conocidas como cargas útiles, con el fin de expandir las funcionalidades del drone Matrice 300 RTK de DJI.
En ese contexto se investigaron distintas maneras de interactuar con dicho drone, siendo algunas el SDK (Software Development Kit) que brinda DJI con el objetivo de darle acceso a los desarrolladores a 


\end{document}